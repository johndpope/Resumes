\documentclass[oneside]{memoir}

\author{Walter Askew}
\title{Application to the Chicago Free School}

\begin{document}

\maketitle

\section*{Why are you interested in working at the Chicago Free
  School?}
After nearly five years as a computer programmer, I find myself ready
to leave the industry for more fulfilling and impactful work.
As I look over my past few years, I can see that my most
rewarding activities have come from teaching opportunities I have
sought outside of work.
My time spent volunteering as a high school tutor, teaching ceramics
classes and instructing informal programming sessions is the time I've
felt the most rewarded by, and I feel myself being tugged into the
teaching profession.

I am interested in the Chicago Free School as a place to grow as an
educator and as vehicle for social justice.
For too many students, school is preparation for incarceration and
class oppression.
I believe that by empowering students in democratic structures that do
not resemble existing systems of oppression, students may find the
space to learn and cultivate hope for a more just future.
I would be proud to join the Chicago Free School and its students in
building a better school together.

\section*{What strengths will you bring to our work to build the
  Chicago Free School?} 
Although I don't have formal education experience, I believe my time
working at a technology startup has provided experience which will be
helpful in growing any aspiring organization.

As a programmer at the startup Narrative Science, I frequently found
myself adapting to roles outside of my engineering position.
While building a product which provides reporting tools for Google
Analytics, I worked on a team of three other programmers which was
lacking project management roles typically available to a product of
this scope.
In addition to the programming work, we were also largely responsible
for planning months of future software development, running status
meetings communicating the project's progress to stakeholders and
organizing our daily tasks together.
The flexibility I learned in adapting into new roles is sure to be
helpful at any startup.

\section*{Starting a new school is very demanding.  What do you
  anticipate will be the most challenging parts of this work and how
  will you respond to these challenges?}
Any startup environment seems prone to difficulties resulting from
lack of resourcing.
Members of a startup frequently find themselves filling needed roles
outside of their strict job description, and the frequent context
switches between roles and lack of resources can become difficult.
The lack of a coherent role can become even more difficult 
given a startup's inherent lack of preexisting structure.
At more established institutions the way forward frequently seems
more clear with years of inertia providing ample push and context,
whereas at a startup culture, vision and structure are still in the
process of being constructed.

As mentioned above, I have experience working in such an environment
and generally enjoy chance to work on a variety of problems and
challenges. 
I am a very grounded person, and I have found my naturally calm
disposition to be a strong resource in chaotic environments.

\section*{What do you look forward to learning as a teacher at the
  Chicago Free School?}
I very much look forward to becoming a stronger educator by learning
from more experienced teachers.
I want to become a professional educator because I find it
very exciting to be at the site of learning and to participate in a
student's growing comprehension.
I hope learn how to become a stronger source of help and support for
young learners.

Equally exciting as the opportunity to become a stronger educator is
the chance to learn about participating in and growing direct
democracy.
The democratic nature of the Chicago Free School has a very exciting
political and social potential, and I am interested in learning how to
grow and develop participation in democratic systems.

\section*{Talk about a lesson or activity you have led as an educator
  that you feel encapsulates some part of your vision for what
  education should be.}
In addition to volunteering as a tutor at the 826CHI, I have
volunteered to help run several workshops hosted at the writing
center.
One of my favorite memories from 826CHI is from a song writing
workshop I helped lead.

I was very excited when the session broke into small groups around
genre and a few students wanted to write a blues song.
I played a few bars of delta blues on my guitar, and the group was
very excited by the foreignness of the sound.
We talked about a few common themes in blues music, and I explained
that a lot of blues music is concerned with a restless desire to
travel and asked the group what it felt like to be trapped someplace
with a desire to move forward.
One girl offered the lyric ``pacing in my mind,'' and I was amazed I
had not heard that line in a blues song before.

This experience writing a blues song with a group of students provides
a vision of what I think education should be, because it contains
elements of cultural exchange, learning through play and mutual
discovery on the part of both the educator and educand.
I was impressed by the students' creative abilities which playfully
burst forth as soon as they were given an outlet, and I was very happy
to be able to share music I love with the students.



\end{document}
