\documentclass[line,margin]{res}
\usepackage[
  colorlinks = true,
  linkcolor = blue,
  urlcolor  = blue,
  citecolor = blue,
  anchorcolor = blue]{hyperref}

\begin{document}

\name{Walter Askew}
% \address used twice to have two lines of address
\address{waltaskew@gmail.com}
\address{404-819-9796}

\begin{resume}
 
\section{ENGINEERING EXPERIENCE}
        {\sl Software Engineer} \hfill May 2014 - July 2014 \\
        Freedom Games, Chicago IL
        \begin{itemize}
          \item Led development of a web application for creation and
            dissemination of K-12 science curricula
          \item Built a REST service to support the application's AngularJS
            front end
            \item Integrated S3 and CloudFront for serving user-uploaded static
              assets
            \item Engineered processes for continuous integration, end-to-end
              browser testing and automated Heroku deployments
            \item Worked as a volunteer at a budding non-profit, designing
              architecture and building the web application's backend in
              Python, Flask and PostgreSQL
        \end{itemize}

        {\sl Software Engineer} \hfill April 2012 - February 2014 \\
        Narrative Science, Chicago IL
        \begin{itemize} 
          \item Planned, built and helped manage the creation of a
            free web application which allows any Google Analytics
            user to sign up for machine-generated reports on their
            websites' traffic and usage patterns
          \item Implemented an automated deployment process for the
            above web application using Amazon Web Services
          \item Designed and implemented a distributed, fault tolerant
            system for continuous ingestion and human verification of
            machine-deduced facts drawn from news headlines
          \item Developed these solutions in Python, using Flask for
            web development
        \end{itemize}

        {\sl Software Engineer} \hfill February 2010 - April 2012 \\
        SteepRock, Inc, Chicago IL
        \begin{itemize}
        \item Named Lead Developer for two of the company's major
          clients, ensured development of client
          deliverables was in line with long-term engineering goals
        \item Developed configurable solutions to meet client
          requirements and be reused as features in the company's
          growing custom web framework
        \item Worked remotely, coding primarily in Python and
          occasionally Javascript
        \end{itemize}

        {\sl Research Assistant} \hfill Summer 2008 \\
        Emory University, Atlanta GA
        \begin{itemize}
        \item Performed Data Mining and Natural Language Processing research
          with Professor Eugene Agichtein and his graduate students,
          eventually submitting a paper and presenting a poster at the National
          Institute of Standards and Technology
        \item Entered national academic challenges focused on data
          mining of medical corpora and recognizing textual entailment
        \end{itemize}

\section{TECHNICAL SKILLS}
        {\sl Languages:} Python, Javascript, C, Java \\
        {\sl Web Frameworks:} Flask, Django \\
        {\sl Operating Systems:} Linux, OS X\\
        {\sl Databases:} Redis, MongoDB, MySQL, PostgreSQL \\
        {\sl Tools:} Git, Bazaar, SVN, CVS, \LaTeX, standard UNIX tools and
        text editors 

\section{EDUCATION}
        {\sl Bachelor of Arts Magna Cum Laude,}
        Computer Science \& Comparative Literature \\
        Emory University \\
        May 2009

\section{PUBLICATIONS}
\begin{itemize}
  \item \href{http://www.nist.gov/tac/publications/2008/participant.papers/Emory.proceedings.pdf}{Combining Lexical, Syntactic, and Semantic Evidence For
      Textual Entailment Classification.  E. Agichtein, W. Askew, Y. Liu
      (TAC 2008)}
  \item \href{https://etd.library.emory.edu/view/record/pid/emory:1b6tn}
    {Predicting Disease Comorbidity by Mining Large Text
      Corpora. Walter Askew (2009)}
\end{itemize}

\end{resume}
\end{document}
